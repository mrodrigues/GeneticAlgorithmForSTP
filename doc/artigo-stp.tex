\documentclass[12pt,a4paper]{article}
\usepackage{hyperref}
\usepackage{titling}
\usepackage{blindtext}
\usepackage{amsmath}
\usepackage[brazil]{babel}
\usepackage[utf8]{inputenc}
\usepackage{natbib}
\usepackage{setspace}
\usepackage{subfig}

\def\foreign#1{%
  \ifcsname TERM#1\endcsname%
  #1%
  \else%
  \emph{#1}%
  \expandafter\gdef\csname TERM#1\endcsname{}%
\fi}

\newcommand{\myinfo}[2]{
  \gdef\@myinfo{%
    \begin{tabular}{ll}
      \textit{E-mail:} & \url{marcosrodrigues@id.uff.br}\\
      \textit{Curso:} & #2\\
   \textit{Professor:} & #1
    \end{tabular}
    \par\vspace{12pt}
  \textsc{Departamento de Ciência da Computação, Universidade Federal Fluminense}}
}
% add the extra information after the date
\postdate{\par\vspace{12pt}\@myinfo\end{center}}

\captionsetup{belowskip=12pt,aboveskip=4pt,labelfont=bf,textfont={bf,it}}

\title{Introduzindo variedade na população inicial de um Algoritmo Genético para
o School Timetabling Problem}
\author{Marcos Felipe P. Rodrigues}
\myinfo{Luiz Satoru Ochi}{Pesquisa Operacional}
\date{\today}

\singlespacing
\begin{document}
\maketitle
\text

\section{Introdução}
O \foreign{School Timetabling Problem} (STP) é um problema discreto,
combinatorial, considerado NP-árduo para quase todas as suas variantes, que
trata do encontro de professores com estudantes, garantindo que certas
restrições e requerimentos sejam cumpridos. Uma solução manual deste problema
levar dias ou até semanas para executar, e normalmente produz resultados
insatisfatórios devido à conflitos com necessidades pedagógicas, entre outros
\cite{Santos:2004:TSSTP}. Dessa forma, é comum na literatura encontrar
metaheurísticas aplicadas com sucesso à esse problema.

Este trabalho busca aprimorar o algoritmo genético de Raghavjee
\cite{Raghavjee:2008:GASTP}, ao uma introduzir aleatoriedade controlada em sua
fase de construção da população inicial. Devido ao alto nível de restrições do
problema, uma construção puramente aleatória demoraria muito para, ou nem mesmo
conseguiria, convergir; por isso, o trabalho anterior utilizou um método que
busca minimizar os conflitos antes de iniciar-se o ciclo de reprodução.
Entretanto, a única variedade produzida nessa população inicial ocorreria ao
tornar-se impossível evitar o conflito; este trabalho apresenta uma alternativa,
em que um candidato é escolhido dentre uma seleção de melhores opções. Todo o
material utilizado neste estudo está hospedado em \cite{Github}, incluindo o
código do algoritmo, dados de entrada e resultados.

\section{Trabalhos Relacionados}
Como as características do STP são altamente dependentes do local onde o
algoritmo será implantado, às vezes variando entre instituições de um mesmo
país, não há um modelo largamente aceito na literatura \cite{Santos:2004:TSSTP}.
Portanto, embora este seja um problema clássico, mesmo algoritmos e heurísticas
semelhantes podem usar modelagens distintas.

Brito et al. \cite{Brito:2012:SAVNS} usa o Kingston's High School Timetabling
Engine (KHE) \cite{Kingston:KHE} para gerar as soluções iniciais, cujas
instâncias representam um formato XML. Em seguida, usa uma metaheurística
Simulated Annealing para aprimorar a solução inicial, e então aplica uma busca
local usando Variable Neighborhood Search. Santos et al.
\cite{Santos:2004:TSSTP} utiliza como modelagem tabelas cujas linhas representam
professores e as colunas, períodos, que simultaneamente impede conflitos de
períodos para os professores e permite a definição de horários indisponíveis em
suas agendas. Sua função objetivo consiste de uma soma de restrições leves e
pesadas, com apropriados pesos para cada. As soluções iniciais são construídas
através de um \foreign{greedy randomized constructive procedure}, e então é
realizada uma Tabu Search, evitando assim que se repitam movimentos recentes.

Raghavjee et al. \cite{Raghavjee:2008:GASTP} utiliza também uma modelagem
baseada em tabelas, mas as linhas são períodos e as colunas, salas de aula, e
neste caso há apenas restrições pesadas. Após uma pesada fase de construção que
garante a alocação de todos os encontros entre professores e alunos, é utilizado
um Genetic Algorithm para buscar uma solução possível. Em seu estudo, Raghavjee
mostra como essa solução foi capaz de se igualar, e até superar, heurísticas
Simulated Annealing, Tabu Search e redes neurais. Devido à estas considerações,
um interesse profissional pela modelagem com salas de aula e um fascínio pessoal
por algoritmos genéticos, este estudo foi a base do presente trabalho.

\section{Definição do Problema}
Seja um conjunto T de professores, um conjunto C de turmas, um conjunto V de
salas de aulas e um número P de períodos semanais. Dada uma lista de
requerimentos definindo a quantidade de períodos semanais que um professor $ t\in T $ 
deve ministrar à turma $ c \in C $ na sala $ v \in V $, realizar o agendamento
atendendo atendendo às restrições. Um agendamento é definido como uma tupla
(t,c,v) alocada para um dos períodos semanais. As restrições podem ser pesadas
ou leves; as primeiras são necessárias para que a solução seja possível, como
não haver conflitos de agendamento, e as segundas são desejáveis, como reduzir o
número de dias da semana que um professor deverá ministrar alguma aula. O
objetivo do algoritmo é encontrar uma solução possível e que atenda o maior
número de restrições leves o possível. Neste estudo, utlizamos apenas restrições
pesadas.

  \subsection{Representação da Solução}
  Cada solução contém duas agendas, uma para as turmas e outra para os
  professores. Cada agenda é representada por uma tabela, cujas linhas são os
  períodos e as colunas são as salas. Para todo $ p \in \{0,1,...,P\} $ e 
  $ v \in V $, o valor $ q_{pv} $ indica a turma ou o professor alocado
  naquela sala, naquele período. Ou seja, alocar uma tupla $ (t,c,v) $ em um período
  $ p $ significa realizar $ q_{pv} = c $ na agenda das turmas e
  $ q_{pv} = t $ na agenda dos professores. Esta representação possui a
  vantagem de não permitir conflitos em salas de aula.

  \subsection{Função Aptidão}
  Neste estudo, são consideradas apenas as restrições pesadas de não haver
  conflito de agendamento entre professores, entre turmas e entre salas de aula.
  Desse modo, a função aptidão é dada pela soma dos conflitos de qualquer tipo
  de uma solução, e deve ser minimizada.

\section{Algoritmo Genético para o School Timetabling Problem}
Este trabalho foi baseado no algoritmo desenvolvido em
\cite{Raghavjee:2008:GASTP}, usando como base o algoritmo genético
disponibilizado em \cite{Clever}, introduzindo aleatoriedade em seus elementos
\foreign{greedy}. Isto ocorre em dois pontos durante a construção da população
inicial: na seleção da melhor tupla para ser alocada, no \foreign{sequential
construction method} (SCM), e na seleção do melhor indivíduo para ser inserido
na população. O algoritmo não utiliza \foreign{crossover}, devido à natureza
altamente restritiva do problema.

  \subsection{Construção}
  Inicialmente, é gerada uma lista de encontros, de acordo com os
  requerimentos, constituída por tuplas (t,c,v), repetidas p vezes, sendo p o
  número de períodos semanais que o professor t deve ministrar à turma c na sala
  v. Dessa forma, a construção da população inicial é feita alocando-se tuplas
  nos períodos mais apropriados.

  A construção é feita executando-se um número n pré-definido de vezes o
  algoritmo SCM. Para cada iteração de zero até n, é gerada uma solução
  candidata, a qual é inserida num conjunto L de candidatas. Ao final,
  constrói-se uma \foreign{restricted candidate list} (RCL), definida por:
  \[
  RCL = \{ l_{ij} \in L | fitness(l) \leq max\_fit(L) - \alpha \times (max\_fit(L) - min\_fit(L)) \}
  \]
  Sendo $ max\_fit $ e $ min\_fit $, respectivamente, as funções que retornam a
  menor e a maior aptidão da lista. O valor $ \alpha $ é o fator de
  \foreign{greediness}, variando de puramente \foreign{greedy} ($ \alpha = 0 $) até
  puramente aleatória ($ \alpha = 1 $).

  O algoritmo SCM classifica as tuplas em ordem decrescente de grau de
  saturação, i.e., o número de períodos nos quais a tupla pode ser alocada sem
  causar conflito. Então, gera uma RCL da mesma forma que definida acima, mas
  trocando as funções fitness, max\_fitness e min\_firtness por saturation\_degree,
  max\_saturation\_degree, min\_saturation\_degree. A melhor tupla é, então,
  inserida na primeira posição disponível em que não vá causar conflito. Após
  a iteração, os graus de saturação devem ser calculados outra vez, uma vez que
  o estado da agenda foi alterado.

  \subsection{Seleção}
  A seleção é feita através de uma variação do \foreign{tournament selection},
  no qual define-se aleatoriamente um vencedor inicial, e para $ i \in \{0,1,...,M\} $,
  sendo M um parâmetro do algoritmo, é realizado um torneio.  Sorteia-se um
  participante da população diferente do vencedor, e sorteia-se um
  número $ d \in \{0,1,2\} $. Caso o número d seja divisível por 3, o combate
  ocorrerá normalmente, e vencerá aquele que possuir o melhor fitness. Caso o
  resultado da divisão de d por 3 seja 1, o participante torna-se o vencedor,
  invariavelmente; caso contrário, o vencedor continua em sua posição,
  invariavelmente. Dessa forma, busca-se simular mais fielmente eventos de um
  torneio real, no qual há um fator sorte envolvido.

  \subsection{Mutação}
  O operador de mutação, primeiro, escolhe aleatoriamente fazer uma alteração na
  tabela dos professores ou das turmas do indivíduo. Então, é selecionada uma
  linha que contenha um conflito. Como um conflito ocorre quando há dois
  professores/turmas num mesmo período, pode-se calcular qual é o
  professor/turma faltante. Deste modo, busca-se uma linha que contenha o
  elemento faltante \textbf{em uma das salas onde houve conflito}, e realiza-se
  a troca. Deste modo, garante-se a integridade dos requerimentos. Caso não seja
  encontrada nenhuma linha apropriada, a troca é realizada em outra qualquer,
  mesmo que não haja conflito. Este processo é realizado um número $Mutation
  Swaps$ de vezes. Caso o filho gerado por essa mutação tenha fitness pior que o
  pai, é descartado e o pai é copiado para a próxima geração.

\section{Configuração}
Os testes foram realizados em uma máquina com 2498MB de RAM, processador AMD de
32 bits de barramento, com dois \foreign{cores} e 1.9Ghz de \foreign{clock}. A
linguagem utilizada foi Ruby versão 2.0.0p247, com o interpretador MRI. Os dados
de entrada foram gerados aleatoriamente a partir dos parâmetros definidos
abaixo, uma vez que não foi possível encontrar os dados do problema original, e
reutilizados em todos os testes. Os parâmetros foram os mesmos que os utilizados
em \cite{Raghavjee:2008:GASTP}, exceto o fator de \foreign{greediness}
introduzido.

\begin{table}
  \caption{Características do Problema}
  \centering
  \begin{tabular}{ c c c c c }
    Problema & Turmas & Professores & Salas & Períodos \\
       hdtt4 & 4 & 4 & 4 & 30 \\
  \end{tabular}
\end{table}

\begin{table}
  \caption{Parâmetros Genéticos}
  \centering
  \begin{tabular}{ c c }
    N (tamanho do SCM) & 1 \\
Tamanho da População & 20, 100 e 1001\\
M (tamanho do torneio) & 10 \\
 No. de Mutation Swaps & 20 \\
  No. Máximo de Gerações & 50 \\
        Fator de Greediness & 0.3 \\
  \end{tabular}
\end{table}

\section{Resultados Computacionais}
A versão modificada foi capaz de melhorar a solução em todos os casos
verificados, em troca de uma performance ligeiramente mais lenta. Apenas com a
população de 1001 indivíduos o algoritmo original conseguiu alcançar o mesmo
melhor resultado que a versão modificada, a qual teve até mesmo uma menor média
de tempo com a população de 100 indivíduos. Os testes foram executados 10 vezes
cada, usando um diferente \foreign{seed} para a geração de números aleatórios a
cada vez.

\begin{table}
  \caption{Média do tempo e do fitness de acordo com a população}
  \centering
  \begin{tabular}{ r r r r r r r }
    Algoritmo & pop size & fitness avg & best fitness & fitness SD & time avg & time SD \\
     original &   20 & 17.6 & 16 & 1.075 &   20.747 &  0.238 \\
   modificado &   20 & 16.4 & 14 & 1.506 &   22.394 &  0.350 \\
     original &  100 & 16.2 & 15 & 0.632 &  116.317 &  4.336 \\
   modificado &  100 & 15.3 & 14 & 0.949 &  113.903 &  2.237 \\
     original & 1001 & 14.4 & 13 & 0.966 & 1037.548 & 20.070 \\
   modificado & 1001 & 13.4 & 13 & 0.516 & 1130.555 & 41.300 \\
  \end{tabular}
\end{table}

\section{Conclusão}
Em nenhum dos casos foi possível chegar à uma solução possível, como foi
atingido em \cite{Raghavjee:2008:GASTP}. Isto pode ser devido à diferenças de
implementação do algoritmo, ou à geração aleatória dos dados de entrada. De
qualquer forma, é necessária maior investigação para que se possa utilizar esta
solução. Apesar disso, os resultados foram melhores após a introdução da
aleatoriedade nas seções \foreign{greedy}, às custas de uma maior lentidão no
algoritmo, o que indica que o original não era capaz de gerar variedade o
suficiente.

Para os próximos passos, seria interessante introduzir outros requerimentos no
algoritmo, de modo a tornar a aplicação mais flexível. Restrições como períodos
indisponíveis para os professores, e requisitos pedagógicos como uma turma não
ter mais de duas aulas com o mesmo professor no mesmo dia, são considerações
comuns no cotidiano das escolas. Também seria de valor otimizar o algoritmo
modificado, de modo que o prejuízo de tempo seja menor.

Por fim, mais testes são necessários, com diferentes dados de entrada. Os dados
coletados são poucos para confirmar que as modificações trouxeram benefícios em
todos os casos. Principalmente, é necessário investigar o motivo de não ter sido
possível alcançar uma solução possível.

\bibliography{research}
\bibliographystyle{abbrv}
\end{document}
